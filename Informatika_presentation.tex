\documentclass[14pt]{beamer}
\usepackage[T2A]{fontenc}
\usepackage[utf8]{inputenc}
\usepackage[english,russian]{babel}
\usepackage{tikz}
\usepackage[european,cuteinductors,smartlabels]{circuitikz}

\usepackage{amssymb,amsfonts,amsmath,mathtext}
\usepackage{amssymb}
\usepackage{cite,enumerate,float,indentfirst}
\usepackage{cancel}
\usepackage{csquotes}
\newcommand{\quotes}[1]{``#1''}

%\usepackage{pgfplots}
%\usepackage[left=1cm,right=1cm, top=1cm,bottom=1cm,bindingoffset=0cm]{geometry}

% Beamer — верстаем презентации  https://habrahabr.ru/post/145523/ 
\graphicspath{{images/}}

\usetheme{Pittsburgh}
\usecolortheme{whale}

\setbeamercolor{footline}{fg=blue}
\setbeamertemplate{footline}{
\leavevmode%
\hbox{%
\begin{beamercolorbox}[wd=.333333\paperwidth,ht=2.25ex,dp=1ex,center]{}%
Прокшин А.Н. и др.
\end{beamercolorbox}%
\begin{beamercolorbox}[wd=.333333\paperwidth,ht=2.25ex,dp=1ex,center]{}%
Санкт-Петербург, 2018
\end{beamercolorbox}%
\begin{beamercolorbox}[wd=.333333\paperwidth,ht=2.25ex,dp=1ex,right]{}%
Стр. \insertframenumber{} из \inserttotalframenumber \hspace*{2ex}
\end{beamercolorbox}}%
\vskip0pt%
}

\newcommand{\itemi}{\item[\checkmark]}

	\usefonttheme[onlymath]{serif} % в формулах использовать текст с засечками
\begin{document}
\title{\small{О системах координат для математического описания систем управления
электропривода.}}
\author{\small{%
\emph{авторы:}~Прокшин Артем Николаевич\\%
\emph{}~Халявин Дмитрий Игоревич\\
\emph{}~Маслов Иван Андреевич\\
\emph{}~Илюшин Антон Геннадьевич\\
\emph{}~Смагин Сергей Игоревич}}
\institute{Санкт-Петербургский государственный электротехнический университет «ЛЭТИ» им. В.И. Ульянова (Ленина)}
\vspace{30pt}%
%Санкт-Петербургский государственный электротехнический университет\\
%«ЛЭТИ» им. В.И. Ульянова (Ленина)

\vspace{60pt}%
\date{5 февраля 2018}
%\date{\small{Санкт-Петербург, 2018}}

\AtBeginSection{
	\begin{frame}
		\frametitle{Содержание}
		\tableofcontents[currentsection]
	\end{frame}
}

\begin{frame}
\titlepage	
\end{frame}

%\begin{frame}
%        \frametitle{Содержание}
%        \tableofcontents[currentsection] 
%\end{frame}

\section{Пространственный вектор в трехфазной системе}
\begin{frame}
\frametitle{\small Пространственный вектор в трехфазной системе}

%\tableofcontents[Пространственный вектор в трехфазной системе]
cимметричная трехфазная система $i_{A} + i_{B} + i_{C} = 0$
\begin{circuitikz}
\newcommand{\D}{4}
	\draw[thick,->,>=stealth'] (0,0) -- node[midway,below] {$\vec{F}_A$} (\D,0) node[above] {$\widetilde{\vec{F}}_1$}; 
	\draw[->,>=stealth'] ({\D/2*cos(240)},{\D/2*sin(240}) node[left] {C} -- ({\D/2*cos(60)},{\D/2*sin(60}) node[above] {$\vec{F}_{1C}$};
	\draw[->,>=stealth'] ({\D/2*cos(120)},{\D/2*sin(120}) node[left] {B} -- ({\D/2*cos(-60)},{\D/2*sin(-60}) node[below] {$\vec{F}_{1B}$};

	\draw[thin] (0,-0.04) -- (\D,-0.04);
	\draw[->,>=stealth'] (\D,0) --++  ({\D/2*cos(-60)},{\D/2*sin(-60})  --++ ({\D/2*cos(60)},{\D/2*sin(60}) ;

	\draw[thin,red,->,>=stealth'] (0,0.04) -- ({3/2*\D},0.04);

	\draw[thin,<-,>=stealth'] ({\D + 1/4*\D},0.04) --++ ({\D/6*cos(60)},{\D/6*sin(60)}) node[above right] {$\Sigma \vec{F}$};
\end{circuitikz}	

$$
\widetilde{\vec{F}}_1 = \only<2>{{\color{red}\frac{2}{3}}} \only<1>{\frac{2}{3}} \left( \vec{F}_{1A} + \vec{F}_{1B} + \vec{F}_{1C} \right)
$$
\end{frame}

\section{проекции пространственного вектора}
\frame{\frametitle{\small проекции пространственного вектора тока в выбранной фазе}
\vspace{-1cm}
\begin{table}%
	\flushleft
\begin{tabular}{@{}p{4.5cm}p{9cm}@{}}% 
\parbox[t]{4.5cm}{
\begin{circuitikz}
	\draw (0,0) to[sV] (0,2.5) to[R,l={$R$}] (2,2.5) to[short, i>=$P\!\sim\!I^2 R$] (4,2.5) -- (4,0) -- (0,0);
\end{circuitikz}
}
	& 
\only<2,3>{
%\parbox[t]{6cm}{
	\begin{circuitikz} 
	\newcommand{\D}{2.5}
	\draw[thin,->,>=stealth'] (-2.5,0) -- (2.5,0);
	\draw[thin,->,>=stealth'] (0,-2) -- (0,2);
	\draw[thick,->,>=stealth',blue] (0,0) -- ({\D*cos(30)}, {\D*sin(30)}) node[above right] {$I$};
	\draw[thin,dashed] ({\D*cos(30)}, {\D*sin(30)}) -- ({\D*cos(30)}, 0) node[below] {$I\cos\omega t$};
	\end{circuitikz}
%}
	} \\
\end{tabular}
\end{table}

\only<3>{
\vspace{1cm}
Измеряемая величина -- \\перпендикулярная проекция вектора $I$ на ось фазы A
}
}
\frame{\frametitle{\small имеющиеся проблемы}
\vspace{-0.7cm} 
\begin{center}
\begin{circuitikz}
	\newcommand{\I}{3}
%	\newcommand{\Y}{3.5}
	\newcommand{\teta}{65}
	\newcommand{\alfa}{25}

        \newcommand{\X}{\I*cos(\alfa)}
        \newcommand{\Y}{\I*sin(\alfa)}
        \newcommand{\Yaxe}{3.5}
        % ось
	\draw[very thin,->,>=stealth'] (0,0) -- (3.5,0) node[right] {$x$};
        \draw[very thin, ->,>=stealth'] (0,0) -- ({\Yaxe*cos(\teta)},{\Yaxe*sin(\teta)}) node[right] {$y$};

	\draw[->,>=stealth'] (0,0) -- ({\I*cos(\alfa)}, {\I*sin(\alfa)}) node[above right] {$\vec{I}$};
        % проекция на ось Х
	\draw[very thin, blue,loosely dashed] ({\X},{\Y}) -- ({\X-\Y*cos(\teta)/sin(\teta)}, 0) node[below] {$\small{\color{black}x_\parallel}$};

	\draw[thin, blue,dotted] ({\X},{\Y}) -- ({\X},0) node[below] {$\small{\color{black}x_\perp}$};
        % проекция на ось Y

	\draw[very thin,blue,loosely dashed] ({\X}, {\Y}) -- ( {\Y*cos(\teta)/sin(\teta)} ,{\Y}) node[left] {$\small{\color{black}y_\parallel}$};;
        % перпендикуляр на ось Y
        \newcommand{\perpend}{\I*cos(\teta-\alfa)}
	\draw[thin,blue, dotted] ({\X}, {\Y}) -- ({\perpend*cos(\teta)},{\perpend*sin(\teta)}) node[above left] {$\small{\color{black}y_\perp}$}; 
\end{circuitikz}
\end{center} 
\vspace{-0.8cm}
\begin{itemize}
	\item косоугольная(криволинейная) система координат
	\item проекции вектора берутся как $x_\perp$
	\item складываем по правилу параллелограмма ($x_\parallel$)
	\item как узнать длину вектора ${\displaystyle \mid I \mid^2 {\color{red}{\huge{\xcancel{=}}}} \sqrt{x^2 +y^2}}$ 
\end{itemize}
}

\section{ковариантные и контрвариантные координаты}
\frame{\frametitle{\small длина вектора, ковариантные и контрвариантные координаты}
}

\frame{\frametitle{\smallосновной и сопряженный базис, разложение по базисам}
}

\section{длина вектора}
\frame{\frametitle{\small длина вектора в косоугольной системе}
}

\frame{\frametitle{\small как определить координаты вектора}
}



\frame{\frametitle{\smallметрический тензор, поднять/опустить индексы}
}

\section{преобразования при смене системы координат}
\frame{\frametitle{\smallпреобразования координат при смене системы координат}
}

\section{тензор взаимоиндукции}
\frame{\frametitle{\smallтензор взаимоиндукции}
}

\frame{\frametitle{\smallучёт индуктивности рассеяния} 
}

\begin{frame}

\end{frame}


\begin{frame}

\end{frame}



%\addcontentsline{toc}{chapter}{Литература}
\section{Литература}
\begin{thebibliography}{3}
	\bibitem{Sokolovsky}Соколовский Г.Г. Электроприводы переменного тока с частотным регулированием: Учебник для студ. высш.учеб. заведений. 
		-- М. Изд. центр «Академия», 2007. - 272 с.		
	\bibitem{Sedov}Седов Л.И. Механика сплошной среды. т.I -- М. Наука, 1994. 
	\bibitem{Dubrovin}Дубровин Б.А., Новиков С.П., Фоменко А.Т. Современная геометрия, Методы и приложения -- М. Наука, 1986.
\end{thebibliography}
\end{document}

